\PassOptionsToPackage{backend=biber}{biblatex}
% \PassOptionsToPackage{hypertexnames=false}{hyperref}
% \documentclass[titlepage,numbers=noenddot,oneside,%
% cleardoublepage=empty,paper=a4,fontsize=11pt,%
% english,%lockflag%
% ]{scrartcl}
\documentclass[draft]{amsart}

\usepackage[USenglish]{babel}
\usepackage{mathtools}
\usepackage{biblatex}
% \usepackage[definetheorems]{hrftex}
% \usepackage[nodayofweek]{datetime}
\usepackage{csquotes}
\usepackage{hyperref}
\usepackage[capitalise,nameinlink]{cleveref}

\addbibresource{ZoteroBibliographyPositiveMassTheoremBachelorThesis.bib}


% just for prototyping
\usepackage{xcolor}


% math stuff
\usepackage{derivative}
% Intervals
\usepackage{interval}
\intervalconfig{
	soft open fences,
	scaled
}


\newtheorem{theorem}{Theorem}
\newtheorem{lemma}[theorem]{Lemma}
\newtheorem{proposition}[theorem]{Proposition}
\newtheorem{corollary}[theorem]{Corollary}
\newtheorem{remark}[theorem]{Remark}
\newtheorem{definition}[theorem]{Definition}




% \usepackage[toc]{appendix}

% \usepackage[euler-digits,euler-hat-accent]{eulervm}
% \usepackage{newpxtext}.
% \ExecuteBibliographyOptions{maxnames=4}

% % upright differential
% \derivset{\odif}{upright=true}

% math punctuation
\newcommand*{\mathcomma}{,}
\newcommand*{\mathfullstop}{.}

% for e.g. easier conversion between \left(...\right) and \big(...\big)
\DeclarePairedDelimiter{\parens}{(}{)}
\let\p\parens % shortcut, lots of "\parens" make the .tex really cluttered

% verbose definitions
\newcommand*{\defines}{\eqqcolon}
\newcommand*{\definedas}{\coloneqq}
\newcommand*{\definedequivalentto}{\vcentcolon\Leftrightarrow}
\newcommand*{\maps}{\colon}
\newcommand*{\isomorphic}{\cong}
\newcommand*{\isomorphismto}{\xrightarrow{\sim}}
\newcommand*{\onetoone}{\xleftrightarrow{1:1}}
\newcommand*{\into}{\hookrightarrow}
\newcommand*{\onto}{\twoheadrightarrow}
\newcommand*{\from}{\leftarrow}
\newcommand*{\bijectionto}{\lhook\joinrel\twoheadrightarrow}

% verbose limits
\newcommand*{\goesto}{\to}
\newcommand*{\goesupto}{\nearrow}
\newcommand*{\goesdownto}{\searrow}

% explanations
% first (optional) argument - size, then explanation, then whats to be explained
% should be used in math mode
\newcommand{\equalto}[3][]{\underset{\scriptstyle\overset{\mkern4mu\scalebox{1}[0.7]{\verteq[#1]}}{\mathclap{#2}}}{#3}}
% accepts vertical relation as first necessary (can be made with vertrelation)
\newcommand{\underrelate}[4][]{\underset{\overset{\csname #1\endcsname{#2}}{\mathclap{#3}}}{#4}}
\newcommand{\overrelate}[4][]{\overset{\underset{\csname #1\endcsname{#2}}{\mathclap{#3}}}{#4}}
\newcommand{\explain}[3][]{\underrelate[#1]{\uparrow}{#2}{#3}}
\newcommand{\undersubset}[3][]{\underrelate[#1]{\vertrelation{\subset}}{#2}{#3}}
\newcommand{\undersupset}[3][]{\underrelate[#1]{\vertrelation{\supset}}{#2}{#3}}


% evaluation
\newcommand{\evaluateat}[2]{\left.#1\right\rvert_{#2}}




% general math stuff
\newcommand*{\reals}{\mathbb{R}}
\newcommand*{\laplacian}{\Delta}
\newcommand*{\boundary}{\partial}
\DeclareMathOperator{\divergence}{div}
\DeclareMathOperator{\trace}{tr}
\DeclarePairedDelimiter{\abs}{\lvert}{\rvert} % absolute value
\DeclarePairedDelimiter{\norm}{\lVert}{\rVert}
\DeclarePairedDelimiterX{\scalarproduct}[2]{\langle}{\rangle}{#1,#2}
\newcommand{\inverse}[1]{{#1}^{-1}}
\newcommand{\tangentspace}[2]{T_{#1}#2}



% specific to the topic
\newcommand{\ext}{\mathrm{ext}} % for the exterior region
\newcommand{\Mend}{M_{\mathrm{end}}} % for an end of M region
\newcommand{\mass}[2]{\mathfrak{m}_{(#1,#2)}} % for the ADM mass

% riemannian geometry
\newcommand{\Ricci}{\mathrm{Ric}} % Ricci curvature tensor
\newcommand{\riemanncurvature}{\mathrm{Rm}} % Riemann curvature tensor

\newcommand{\todomark}{%
    \colorbox{purple}{%
        \textnormal\ttfamily\bfseries\color{white}%
        TODO%
    }%
}
\newcommand{\todo}[1][]{%
    \ifstrempty{#1}{%
        \def\todotext{Todo}%
    }{%
        \def\todotext{Todo: #1}%
    }%
    \todomark\,%
    {%
        \marginpar{%
            \raggedright\normalfont\sffamily\scriptsize\todotext%
        }%
    }%
}


% turn of hyperlinking inside e.g. captions to avoid bookmarks and stuff
\newcommand*{\nolink}[1]{%
  {\protect\NoHyper#1\protect\endNoHyper}%
}


\title[An equality version of Stern's integral inequality]{A (non-harmonic) equality version of the Stern Integral Inequality and Consequences for Mass}
\vspace{1em}
\author{Henry Ruben Fischer}
\begin{document}

\selectlanguage{USenglish}
\setquotestyle{USenglish}
\maketitle
\definecolor{MyRed}{HTML}{AA2E00}
\definecolor{MyPurple}{HTML}{9011C2}
\definecolor{MyBlue}{HTML}{0905FC}
\definecolor{MyTeal}{HTML}{1368A0}
\newcommand{\nonHarmonic}[1]{\textcolor{teal}{#1}}
\newcommand{\additionalTerms}[1]{\textcolor{red}{#1}}
\newcommand{\additionalNonHarmonicTerms}[1]{\textcolor{purple}{#1}}
\pagenumbering{arabic}
\section{Introduction}
In \cite{brayHarmonicFunctionsMass2019}, Bray, Kazaras, Khuri and Stern used an approach based on level sets of harmonic functions to prove the Riemannian positive mass theorem in \( 3 \) dimensions. The approach centrally employs a certain integral inequality (\cite[Proposition 4.2]{brayHarmonicFunctionsMass2019}), a version of which was first established by Stern in \cite{stern2022scalar}. It is well known (cf.~\cite[Section 4.1]{hirschSpacetimeHarmonicFunctions2021}) that in the case where the harmonic function of interest has no critical points, this inequality becomes an equality. We will prove that equality also holds in the general case, i.e., even in the presence of critical points. By keeping track of an additional term involving the Euler characteristic of level sets, one can then strengthen the lower bounds for the mass obtained in \cite{brayHarmonicFunctionsMass2019} and \cite{hirschSpacetimeHarmonicFunctions2021} into equalities. We will do this explicitly for \cite{brayHarmonicFunctionsMass2019}.

\todo[This is very similar to the third paragraph of \cite{hirschSpacetimeHarmonicFunctions2021}. In general this whole intro is pretty rough right now.] The approach of \cite{brayHarmonicFunctionsMass2019} can be generalized by replacing harmonic functions with other geometrically motivated elliptic equations, allowing to, e.g., prove the positive energy theorem for asymptotically flat initial data sets. For the sake of generality we thus follow \cite{hirschSpacetimeHarmonicFunctions2021} and do not assume harmonicity of our function for now.

\section{The main identity}
\todo[Probably shouldn't keep this if I want to upload this to the ArXiv. Though maybe it's useful to readers just wanting to see what's different to before?] Everything below is pretty much directly adopted from my Bachelor's Thesis (which more or less just followed \cite{hirschSpacetimeHarmonicFunctions2021}), though I have 
\begin{enumerate}
    \item changed \nonHarmonic{some stuff} (marked in \nonHarmonic{blue}) to allow for non harmonic functions \( u \), and
    \item kept track of some \additionalTerms{additional terms} (marked in \additionalTerms{red}) to strengthen the encountered inequalities into equalities.
\end{enumerate}
If something is related to both, it'll be in \additionalNonHarmonicTerms{purple}.

 
% Our main tool motivating our use of harmonic functions is the following integral inequality relating scalar curvature to derivatives of harmonic functions.
% The following is adopted from \cite[Proposition 3.2]{hirschSpacetimeHarmonicFunctions2021} (but we also allow a component of the boundary with \( \abs{\nabla u}=0 \) as in \cite[Proposition 4.2]{brayHarmonicFunctionsMass2019}), and we try to make no assumptions on the harmonicity of \( u \). % or as late as possible?
\begingroup
\newcommand{\maxu}{\overline{u}}
\newcommand{\minu}{\underline{u}}
\newcommand{\nonzeroboundary}{\partial_{\neq 0}\Omega}
\begin{proposition}\label{prop:main_equality}
    Let \( (\Omega,g) \) be a compact oriented Riemannian manifold with boundary having outward unit normal \( \nu \). Let \( u\maps \Omega\to \reals \) be any 
    % harmonic
    smooth function
    % (\ie \( \laplacian u=0 \)) 
    % such that the set of regular points of \( u \) is open and 
    such that there exists \( C_0>0 \) with 
    \begin{equation}
        \abs{\laplacian u}\leq C_0 \cdot \abs{\nabla u}\label{eq:assumption_on_laplacian}\mathfullstop
    \end{equation}
    Denote by \( \nonzeroboundary \) the open set of points in \( \boundary{\Omega} \) with \( \abs{\nabla u}\neq 0 \). If \( \maxu \) and \( \minu \) denote the maximum and minimum of \( u \) and \( S_t \) are \( t \)-level sets of \( u \), then
    \begin{equation*}
        \int_{\minu}^{\maxu}\int_{S_t}\frac{1}{2}\p*{\frac{\abs{\nabla^2 u}^2-\nonHarmonic{(\laplacian u)^2}}{\abs{\nabla u}^2}+R_{\Omega}-R_{S_t}}\odif{A}\odif{t}=\int_{\nonzeroboundary}\p*{\partial_\nu \abs{\nabla u}-\nonHarmonic{\laplacian u \frac{\scalarproduct{\nu}{\nabla u}}{\abs{\nabla u}}}}\odif{A}\mathcomma
    \end{equation*} 
    where \( R_{\Omega} \) and \( R_{S_T} \) denote the scalar curvature of \( \Omega \) and the level sets \( S_t \) respectively.
\end{proposition}

\newcommand{\secondfundamentalform}{\Romannum{2}}
For the proof we follow the \cite[Proof of Proposition 4.2 in][]{brayHarmonicFunctionsMass2019} and use Bochner's identity as well as the following Lemma:
\begin{lemma}\label{lem:traced_gauss} 
    For \( u\maps \Omega\to \reals \) as above with regular level set \( S \) we have
    \begin{equation*}
        2\Ricci(\nabla u, \nabla u)=\abs{\nabla u}^2(R_\Omega-R_S)+2\abs{\nabla\abs{\nabla u}}^2-\abs{\nabla^2 u}^2-\nonHarmonic{2\laplacian u \nabla^2_{\nu\nu}u+(\laplacian u)^2}\mathfullstop
    \end{equation*}
\end{lemma}
\begin{proof}  
    Let \( S \) be a regular level set of \( u \) with induced metric \( \gamma \), second fundamental form \( A_{ij} \) and mean curvature \( H \). The normal to \( S \) is then \( \nu^i=\nabla^i u/\abs{\nabla u} \) and we have
    \begin{equation*}
        \begin{aligned}[t]
            A_{ij}&=\gamma^{k}_i\gamma^l_j \nabla_k (\nabla_l u/\abs{\nabla u})\\
            &=\gamma^{k}_i \gamma^{l}_j\frac{\nabla^2_{kl}u}{\abs{\nabla u}}+(\dotsc)\gamma^l_j\nabla_l u \\
            &=\gamma^{k}_i \gamma^l_j \frac{\nabla^2_{kl}u}{\abs{\nabla u}}\\
            &=(g^k_i-\nu_i \nu^k)(g^l_j - \nu^l \nu_j)\frac{\nabla^2_{kl}u}{\abs{\nabla u}}\\
            &=\frac{\overbrace{\nabla^2_{ij}u}^{    \text{Term \( T^1 \)}}-\overbrace{\nu_i \nu^k\nabla^2_{kj}u}^{T^2}-\overbrace{\nu_j \nu^l\nabla^2_{il}u}^{T^3}+\overbrace{\nu_i \nu_j \nu^k \nu^l\nabla_{kl}u}^{T^4}}{\abs{\nabla u}}
        \end{aligned}
    \end{equation*}
    and thus (below we use c.w.~to denote which term \enquote{contracted with} which other term)
    \newcommand{\cw}{\text{~c.w.~}}
    \begin{equation*}
        \begin{aligned}[t]
            \abs{A}^2&=\begin{aligned}[t]
                &\frac{\overbrace{\abs{\nabla^2 u}^2}^{T^1\cw T^1}+\overbrace{(\nabla^2_{\nu\nu}u)^2}^{T^4\cw T^4}}{\abs{\nabla u}^2}\\
            &+\frac{\overbrace{2\nu^k \nu_l \nabla^2_{kj}u (\nabla^2)^{lj}u}^{T^{2/3}\cw T^{2/3}}-\overbrace{4\nabla^2_{ij}u \nu^i \nu_k (\nabla^2)^{kj}u}^{T^{1}\cw T^{2/3}}}{\abs{\nabla u}^2}\\
            &+\frac{\overbrace{2(\nabla^2_{\nu\nu}u)^2}^{T^{2/3}\cw T^{3/2}}-\overbrace{4(\nabla^2_{\nu\nu}u)^2}^{T^{2/3}\cw T^{4}}+\overbrace{2(\nabla^2_{\nu\nu}u)^2}^{T^{1}\cw T^{4}}}{\abs{\nabla u}^2}
            \end{aligned}\\
            &=\frac{\abs{\nabla^2 u}^2-2g^{jk}\nabla^2_{kl}u\nabla^l u\nabla^2_{ij}u \nabla^i{u}/(\abs{\nabla u}^2)+(\nabla^2_{\nu\nu}u)^2}{\abs{\nabla u}^2}\\
            &=\frac{\abs{\nabla^2 u}^2-(\nabla_k(\nabla_l u \nabla^l u)\nabla^k(\nabla_l \nabla^l u))/(2\abs{\nabla u}^2) +(\nabla^2_{\nu\nu}u)^2}{\abs{\nabla u}^2}\\
            &=\frac{\abs{\nabla^2 u}^2-\abs{\nabla\abs{\nabla u}^2}^2/(2\abs{\nabla u}^2)+(\nabla^2_{\nu\nu}u)^2}{\abs{\nabla u}^2}\\
            &=\frac{1}{\abs{\nabla u}^2}(\abs{\nabla^2 u}^2-2\abs{\nabla \abs{\nabla u}}^2+(\nabla^2_{\nu\nu}u)^2)\mathfullstop
        \end{aligned}
    \end{equation*}
    On the other hand contracting \( A_{ij} \) gives
    \begin{equation*}
        H=\frac{1}{\abs{\nabla u}}(\nonHarmonic{\laplacian u}
        -\nabla^2_{\nu \nu} u)\mathfullstop
    \end{equation*}
    and thus
    \begin{equation*}
        \abs{A}^2-H^2=\abs{\nabla u}^{-2}(\abs{\nabla^2 u}^2-2\abs{\nabla\abs{\nabla u}}^2+\nonHarmonic{2\laplacian u \nabla^2_{\nu\nu}u-(\laplacian u)^2})\mathfullstop
    \end{equation*}
    Combining with the contracted Gauss-Codazzi equation
    \begin{equation*}
        2\Ricci((\nabla u)/\abs{\nabla u},(\nabla u)/\abs{\nabla u})=R_\Omega-R_S+H^2-\abs{A}^2\mathcomma
    \end{equation*} 
    then yields the result.
\end{proof}


\begin{proof}[Proof of \cref{prop:main_equality}]
    During the following proof, we will be considering 
    \begin{equation*}
        \varphi_\varepsilon\definedas \sqrt{\abs{\nabla u}^2+\varepsilon} 
    \end{equation*}
    for \( \varepsilon>0 \) instead of \( \abs{\nabla u} \), since we cannot control the behavior of integrands like \( \laplacian \abs{\nabla u} \) and \( \partial_\nu \abs{\nabla u} \) at critical points of \( u \) (where \( \abs{\nabla u}=0 \)).

    Recall first \emph{Bochner's identity},
    \begin{equation*}
        \frac{1}{2}\laplacian(\abs{\nabla u}^2)=\abs{\nabla^2 u}^2+\nonHarmonic{\scalarproduct{\nabla \laplacian u}{\nabla u}}+\Ricci(\nabla u, \nabla u)\mathfullstop
    \end{equation*} 

    We find
    \begin{equation}\label{eq:main_equality_applied_bochner}
        \begin{aligned}[b]
            \laplacian \varphi_\varepsilon&=\nabla_i \nabla^i \sqrt{\abs{\nabla u}^2+\varepsilon}\\
            &=\nabla_i \frac{\nabla^i \abs{\nabla u}^2}{2\varphi_\varepsilon}\\
            &=\frac{\laplacian \abs{\nabla u}^2}{2\varphi_\varepsilon}-\frac{\abs{\nabla\abs{\nabla u}^2}^2}{4\varphi_\varepsilon^3}\\
            &=\varphi_{\varepsilon}^{-1}(\abs{\nabla^2 u}^2+\Ricci(\nabla u,\nabla u)-\abs{\nabla \abs{\nabla u}^2}^2/(4\varphi_\varepsilon^2)+\nonHarmonic{\scalarproduct{\nabla \laplacian u}{\nabla u}})\mathcomma
        \end{aligned}
    \end{equation}
    where we have used Bochner's identity on the last line.

    On a regular level set \( S \), \cref{lem:traced_gauss} thus yields
    \begin{equation}
        \laplacian \varphi_{\varepsilon}=\begin{aligned}[t]
            \frac{1}{2\varphi_{\varepsilon}}\Bigl(&\abs{\nabla^2 u}^2+\abs{\nabla u}^2(R_\Omega-R_S)\\
            &+\additionalTerms{2 \cdot (1-\varphi_\varepsilon^{-2}\abs{\nabla u}^2)\abs{\nabla \abs{\nabla u}}^2}\\
            &+\nonHarmonic{2\scalarproduct{\nabla (\laplacian u)}{\nabla u}+(\laplacian u)^2-2(\laplacian u)\nabla^2_{\nu\nu}u}\Bigr)
        \end{aligned}\mathfullstop\label{eq:main_equality_applied_main_identity}
    \end{equation}
    Note that
    \nonHarmonic{
    \begin{equation}
        \begin{aligned}
            &\frac{(\laplacian u)^2}{\varphi_{\varepsilon}}+\frac{\scalarproduct{\nabla u}{ \nabla \laplacian u}}{\varphi_{\varepsilon}}-\divergence\p*{\laplacian u \frac{\nabla u}{\varphi_\varepsilon}}\\
        &\qquad=\frac{\laplacian u}{\varphi_\varepsilon^2} \cdot \scalarproduct{\nabla u}{\nabla \varphi_{\varepsilon}}=\frac{\laplacian u}{2 \varphi_\varepsilon^3} \cdot \scalarproduct{\nabla u}{\nabla \abs{\nabla u}^2}\mathfullstop\label{eq:divergence_stepping_stone}
        \end{aligned}
    \end{equation}}
    and thus one obtains in general
    \begin{equation}
        \divergence\p*{\nabla \varphi_{\varepsilon}-\nonHarmonic{\laplacian u \frac{\nabla u}{\varphi_{\varepsilon}}}}=\begin{aligned}[t]
            \frac{1}{\varphi_{\varepsilon}}\biggl(&\abs{\nabla^2 u}^2+\Ricci(\nabla u,\nabla u)-\abs{\nabla\abs{\nabla u}^2}/\p{4\varphi_\varepsilon^2}\\
            &-\nonHarmonic{(\laplacian u)^2+\laplacian u\scalarproduct{\nabla u}{\nabla \abs{\nabla u}^2}/(2\varphi_{\varepsilon}^2)}\biggr)\mathfullstop\label{eq:general_divergence}
        \end{aligned}
    \end{equation}
    On a regular level set, note that
    \begin{equation*}
        \nonHarmonic{ \scalarproduct{\nabla u}{\nabla \abs{\nabla u}^2}=2\nabla^i u\nabla^
        j u \nabla^2_{ij}u=2\abs{\nabla u}^2 \nabla^2_{\nu \nu}u}
    \end{equation*} 
    and combine \cref{eq:divergence_stepping_stone} with \cref {eq:main_equality_applied_main_identity} to get
    \begin{equation}
        \divergence\p*{\nabla \varphi_{\varepsilon}-\nonHarmonic{\laplacian u \frac{\nabla u}{\varphi_{\varepsilon}}}}=\begin{aligned}[t]
            \frac{1}{2\varphi_{\varepsilon}}\Bigl(&\abs{\nabla^2 u}^2+\abs{\nabla u}^2(R_\Omega-R_S)-\nonHarmonic{(\laplacian u)^2}\\
            &+\additionalTerms{2 (1-\abs{\nabla u}^2/(\varphi_{\varepsilon}^2))(\abs{\nabla\abs{\nabla u}}^2-\additionalNonHarmonicTerms{(\laplacian u)\nabla^2_{\nu \nu}u})}\Bigr)\mathfullstop\label{eq:regular_level_set_divergence}
        \end{aligned}
    \end{equation}
    % For convenience, we also define
    % \begin{equation*}
    %     X\definedas  \nabla \varphi_{\varepsilon}-\nonHarmonic{\laplacian u \frac{\nabla u}{\varphi_{\varepsilon}}}.
    % \end{equation*} 
    Let now \( \mathcal{A}\subset \interval{\minu}{\maxu} \) be an open set containing all the critical values of \( u \), and let \( \mathcal{B}=\interval{\minu}{\maxu}\setminus \mathcal{A} \) be the complementary set. Then the divergence theorem yields
    \begin{equation}
        % \int_{\boundary{\Omega}}\scalarproduct{\nu}{X}\odif{A}=\int_{\Omega}\divergence X\odif{V}\mathfullstop\label{eq:main_equality_branching_point}
        \int_{\boundary{\Omega}}\p*{\partial_\nu \varphi-\nonHarmonic{\laplacian u \frac{\scalarproduct*{\nu}{\nabla u}}{\varphi_{\varepsilon}}}}\odif{A}=\int_{\Omega}\divergence\p*{\nabla \varphi_{\varepsilon}-\nonHarmonic{\laplacian u \frac{\nabla u}{\varphi_{\varepsilon}}}}\odif{V}\mathfullstop\label{eq:main_equality_branching_point}
    \end{equation}
    We will deal with the right side of the equation by treating integrals over the preimages of \( \mathcal{A} \) and \( \mathcal{B} \) seperately. 
    
    Let us first consider the integral \( \inverse{u}(\mathcal{A}) \). For convenience, in the following equations \( C \) will always refer to some nonnegative constant (independent of \( \varepsilon \)), but each appearance of \( C \) may denote a different constant. Note that
    \begin{equation*}
        \nonHarmonic{\abs{\laplacian u \scalarproduct{\nabla u}{\nabla \abs{\nabla u}^2}}\explain{\text{Cauchy-Schwarz and \cref{eq:assumption_on_laplacian}}}{\leq} C\cdot \abs{\nabla u}\cdot \abs{\nabla u}\cdot \abs{\nabla \abs{\nabla u}^2}\explain{\text{Young's inequality with \( \varepsilon=1/2 \)}}{\leq}C\cdot \abs{\nabla u}^4+\frac{\abs{\nabla \abs{\nabla u}^2}^2}{8}},
    \end{equation*}
    which then yields
    \begin{align*}
        &\abs{\nabla^2 u}^2-\frac{\abs{\nabla \abs{\nabla u}^2}^2}{4\varphi_\varepsilon^2}+\nonHarmonic{\frac{\laplacian u \scalarproduct{\nabla u}{\nabla \abs{\nabla u}^2}}{2\varphi_{\varepsilon}^2}}\\
        &\qquad \geq -C\cdot \abs{\nabla u}^2+\abs{\nabla^2 u}^2-\frac{5}{4}\cdot \frac{\abs{\nabla\abs{\nabla}^2}^2}{4}\geq -C \cdot \abs{\nabla u}^2,
    \end{align*}
    where the last step is possible due to the refined Kato's inequality \cite[Eq. (C.2)]{braySpacetimeHarmonicFunctions2021}. Together with \cref{eq:general_divergence}, one gets
    \begin{equation*}
        \divergence\p*{\nabla \varphi_{\varepsilon}-\nonHarmonic{\laplacian u \frac{\nabla u}{\varphi_{\varepsilon}}}}\geq \frac{1}{\varphi_{\varepsilon}}(\overbrace{\Ricci(\nabla u,\nabla u)}^{\mathclap{{}\geq -\abs{\Ricci}\abs{\nabla u}^2}}-\nonHarmonic{\underbrace{(\laplacian u)^2}_{\mathclap{\leq C\abs{\nabla u}^2}}}{}- C\cdot \abs{\nabla u}^2)\geq -C\cdot \abs{\nabla u}^2,
    \end{equation*}
    where we have used \cref{eq:assumption_on_laplacian} and that \( \abs{\Ricci} \) is bounded since \( \Omega \) is compact. In particular we can apply the coarea formula and obtain
    \begin{equation}
            -\int_{\inverse{u}(\mathcal{A})}\divergence\p*{\nabla \varphi_{\varepsilon}-\nonHarmonic{\laplacian u \frac{\nabla u}{\varphi_{\varepsilon}}}}\odif{V}\leq \int_{\inverse{u}(\mathcal{A})}C\abs{\nabla u}\odif{V}= C\int_{t\in \mathcal{A}}\mathcal{H}^2(S_t)\odif{t}\mathcomma\label{eq:inequality_near_critical_points}
    \end{equation}
    where \( \mathcal{H}^2(S_t) \) is the Hausdorff measure of the level sets.


    On \( \inverse{u}(\mathcal{B}) \) on the other hand we can apply the coarea formula directly to \cref{eq:regular_level_set_divergence}, which produces
    \begin{equation}
        \begin{aligned}            
            &\int_{\inverse{u}(\mathcal{B})}\divergence\p*{\nabla \varphi_{\varepsilon}-\nonHarmonic{\laplacian u \frac{\nabla u}{\varphi_{\varepsilon}}}}\odif{V}\\
            &\qquad=\frac{1}{2}\int_{t\in \mathcal{B}}\int_{S_t}\frac{\abs{\nabla u}}{2\varphi_{\varepsilon}}\begin{aligned}[t]
                \biggl(&\frac{\abs{\nabla^2 u}^2-\nonHarmonic{(\laplacian u)^2}}{\abs{\nabla u}^2}+R_{\Omega}-R_{S_t}\\
                &+\additionalTerms{2\cdot (\abs{\nabla u}^{-2}-\varphi_{\varepsilon}^{-2})(\abs{\nabla \abs{\nabla u}}^2 -\additionalNonHarmonicTerms{\laplacian u \nabla^2_{\nu \nu}})}\biggr)\odif{A}\odif{t}\mathfullstop
            \end{aligned}\label{eq:away_from_critical_points}
        \end{aligned}
    \end{equation}
    Note that 
    \begin{equation*}
        \partial_\nu \varphi_\varepsilon=\frac{\nu^i \nabla_{ij}u \nabla^j u}{\varphi_\varepsilon}=0
    \end{equation*} 
    at critical points of \( u \) and thus we may replace \( \boundary{\Omega} \) by \( \nonzeroboundary \) in the integral in \cref{eq:main_equality_branching_point} and below in \cref{eq:main_equality_nearly_there}. 
    
    We combine \cref{eq:away_from_critical_points} with \cref{eq:main_equality_branching_point} and obtain
    \begin{equation*}
        \begin{aligned}[t]
        &&\span\frac{1}{2}\int_{t\in \mathcal{B}}\int_{S_t}\frac{\abs{\nabla u}}{\varphi_\varepsilon}\p*{\frac{\abs{\nabla^2 u}^2-\nonHarmonic{(\laplacian u)^2}}{\abs{\nabla u}^2}+R_{\Omega}-R_{S_t}}\odif{A}\odif{t}\\
        &&\span-\int_{\nonzeroboundary}\p*{\partial_\nu \varphi_{\varepsilon}-\nonHarmonic{\laplacian u \frac{\scalarproduct{\nu}{\nabla u}}{\varphi_{\varepsilon}}}}\odif{A}\\
            &&\qquad=&-\int_{\inverse{u}(\mathcal{a})}\divergence\p*{\laplacian \varphi_{\varepsilon}-\nonHarmonic{\laplacian  \frac{\nabla u}{\varphi_{\varepsilon}}}}\odif{V}\\
            &&&{}-\additionalTerms{\int_{t\in \mathcal{B}}\int_{S_t}(\abs{\nabla u}^{-2}-\varphi_{\varepsilon}^{-2})\cdot (\abs{\nabla \abs{\nabla u}}^2 -\additionalNonHarmonicTerms{\laplacian u \nabla^2_{\nu \nu}})\odif A\odif t} \mathfullstop
         \end{aligned}
    \end{equation*}
    Taking absolute values and using \cref{eq:inequality_near_critical_points} yields
    \begin{equation}\label{eq:main_equality_nearly_there}
        \begin{aligned}[t]
        &\Bigg\lvert\frac{1}{2}\int_{t\in \mathcal{B}}\int_{S_t}\frac{\abs{\nabla u}}{\varphi_\varepsilon}\p*{\frac{\abs{\nabla^2 u}^2-\nonHarmonic{(\laplacian u)^2}}{\abs{\nabla u}^2}+R_{\Omega}-R_{S_t}}\odif{A}\odif{t}\\
        &{}-\int_{\nonzeroboundary}\p*{\partial_\nu \varphi_{\varepsilon}-\nonHarmonic{\laplacian u \frac{\scalarproduct{\nu}{\nabla u}}{\varphi_{\varepsilon}}}}\odif{A} \Bigg\rvert\\
            &\qquad\leq C\int_{t\in \mathcal{A}} \mathcal{H}^2(S_t)\odif{t}-\additionalTerms{\int_{t\in \mathcal{B}}\int_{S_t}(\abs{\nabla u}^{-2}-\varphi_{\varepsilon}^{-2})\cdot (\abs{\nabla \abs{\nabla u}}^2 -\additionalNonHarmonicTerms{\laplacian u \nabla^2_{\nu \nu}})\odif A\odif t }\mathfullstop
         \end{aligned}
    \end{equation}

    
    Since \( \Omega \) is compact and \( \mathcal{B} \) closed, \( \abs{\nabla u} \) and \( \abs{\nabla\abs{\nabla u}}^2 \) are uniformly bounded from below on \( \inverse{u}(\mathcal{B}) \). Furthermore on \( \nonzeroboundary \) (where \( \abs{\nabla u}\neq 0 \)) we have
    \begin{equation*}
        \partial_\nu \varphi_\varepsilon=\frac{\abs{\nabla u}}{\varphi_\varepsilon}\partial_{\nu}\abs{\nabla u}\goesto \partial_{\nu}\abs{\nabla u}\quad \text{as \( \varepsilon\goesto 0 \)}
    \end{equation*} 
    We can thus now take the limit \( \varepsilon\goesto 0 \) in \cref{eq:main_equality_nearly_there} and get
    \begin{multline}\label{eq:equality_epsilon_to_zero}
        \begin{aligned}[t]
        &\Bigg\lvert\frac{1}{2}\int_{\mathcal{B}}\int_{S_t}\p*{\frac{\abs{\nabla^2 u}^2-\nonHarmonic{(\laplacian u)^2}}{\abs{\nabla u}^2}+R_{\Omega}-R_{S_t}}\odif{A}\odif{t}\\
        &{}-\int_{\nonzeroboundary}\p*{\partial_\nu \abs{\nabla u}-\nonHarmonic{\laplacian u \frac{\scalarproduct{\nu}{\nabla u}}{\abs{\nabla u}}}}\odif{A}\Bigg\rvert\leq C\int_{t\in \mathcal{A}} \mathcal{H}^2(S_t)\odif{t} \mathfullstop
         \end{aligned}
    \end{multline}
    

    By Sard's theorem (\cite{sardMeasureCriticalValues1942}), the set of critical values has measure \( 0 \) and we thus may take the measure of \( \mathcal{A} \) to be arbitrarily small. Since \( t\mapsto \mathcal{H}^2(S_t) \) is integrable by the coarea formula, taking \( \abs{\mathcal{A}}\to 0 \) in \cref{eq:equality_epsilon_to_zero} leads to
    \begin{equation*}
        \int_{\minu}^{\maxu}\int_{S_t}\frac{1}{2}\p*{\frac{\abs{\nabla^2 u}^2-\nonHarmonic{(\laplacian u)^2}}{\abs{\nabla u}^2}+R_{\Omega}-R_{S_t}}\odif{A}\odif{t}=\int_{\nonzeroboundary}\p*{\partial_\nu \abs{\nabla u}-\nonHarmonic{\laplacian u \frac{\scalarproduct{\nu}{\nabla u}}{\abs{\nabla u}}}}\odif{A}\mathcomma
    \end{equation*} 
    and we have thus proven our claim. 
\end{proof}
\begin{remark}
    Note that the requirement on \( u  \) (namely that \( \abs{\laplacian u}\leq C\cdot \abs {\nabla u} \)) can be weakened a bit: Really one just needs that there exists some \( C \) such that \( -C\abs{\nabla u} \) bounds the RHS of \cref{eq:general_divergence} from below.
\end{remark}
\newcommand{\spacetimehessian}{\bar{\nabla}^2}
\section{Application to (spacetime) harmonic functions}
Let us now apply \cref{prop:main_equality} to the case of spacetime harmonic functions on initial data sets for the Einstein equations to obtain an equality version of \cite[Proposition 4.2]{hirschSpacetimeHarmonicFunctions2021}. 
% The following corollary will serve both as a sanity check for \cref{prop:main_equality} (to see whether it is in the right form) and as a preparation for our proof of an equality version of the lower bound for the mass (of asymptotically flat IDS) achieved in \cite{hirschSpacetimeHarmonicFunctions2021}.

We first recall some definitions (see also \cite[Definition 7.16]{leeGeometricRelativity2019}):
\begin{definition}\label{def:initial_data_set}
    An \emph{initial data set} (IDS) \( (M,g,k) \) is a Riemannian manifold \( (M,g) \) equipped with a symmetric \( 2 \)-tensor \( k \), representing an embedded spacelike hypersurface in \( (3+1) \)-dimensional spacetime, where \( g \) is the induced metric on the hypersurface (also called first fundamental form) and \( k \) is the extrinsic curvature tensor (or second fundamental form). These objects satisfy the \emph{constraint equations}
    \begin{equation}\label{eq:spacetime_definitions}
        J=\divergence(k-(\trace{k})g), \qquad  \mu=\frac{1}{2}(R+(\trace k)^2-\abs{k}^2)\mathfullstop
    \end{equation}
    Here \( R \) is the scalar curvature and \( \mu \) and \( J \) represent energy and momentum density respectively and \( \spacetimehessian \) is called the \emph{spacetime Hessian}.

    We say the IDS fulfills the \emph{dominant energy condition} if \( \mu\geq \abs{J} \).
\end{definition}
\begin{definition}
    Let \( u\maps M\to \reals \) be a smooth function. We define the \emph{spacetime hessian},
    \begin{equation*}
        \spacetimehessian u \definedas\nabla^2 u+k\abs{\nabla u}\mathcomma
    \end{equation*}
    and call \( u \) a \emph{spacetime harmonic function} if it satisfies the equation
    \begin{equation*}
        \trace{\spacetimehessian u}=0,
    \end{equation*}
    i.e. if
    \begin{equation*}
        \laplacian u=-\trace(k)\abs{\nabla u}.
    \end{equation*}
\end{definition}
\begin{corollary}[Equality version of \texorpdfstring{}{\nolink}{\cite[Proposition 4.2]{hirschSpacetimeHarmonicFunctions2021}}]
    Let \( (\Omega,g,k) \) be an oriented compact initial data set (IDS) with smooth boundary \( \boundary{\Omega} \), having outward unit normal \( \nu \). Let \( u\maps \Omega\to \reals \) be a spacetime harmonic function, and let \( \nonzeroboundary \), \( \minu \), \( \maxu \), \( S_t \) and \( K \) be as in \cref{prop:main_equality}. Then
    \begin{equation*}
        \int_{\nonzeroboundary} (\partial_\nu \abs{\nabla u}+k(\nabla u,\nu))\odif{A}=\int_{\minu}^{\maxu}\int_{S_t} \p*{\frac{1}{2}\frac{\abs{\spacetimehessian u}}{\abs{\nabla u}^2}+\mu+J(\nabla u/\abs{\nabla u})-K}\odif A\odif{t},
    \end{equation*}
    where \( K \) is the level set Gauss curvature.
\end{corollary}
Note that the original \cite[Proposition 4.2]{hirschSpacetimeHarmonicFunctions2021} uses \( \nu \) differently than we do. In the above, \( \nu \) is still the normal vector of the level sets. 
\begin{proof}
    Let \( C_0 \) be any constant bounding \( \abs{\trace{k}} \). Then since \( u   \) is spacetime harmonic, we have \( \abs{\laplacian u}\leq C_0\cdot \abs{\nabla u} \) and can thus apply \cref{prop:main_equality} to get
    \begin{equation*}
        \int_{\minu}^{\maxu}\int_{S_t} \frac{1}{2}\p*{\frac{\abs{\nabla^2 u}}{\abs{\nabla u}^2}-(\trace k)^2+R-2K }\odif{A}\odif{t}=      \int_{\nonzeroboundary }\p*{\partial_{\nu} \abs{\nabla u}+(\trace k)\scalarproduct{\nu}{\nabla u}}\odif{A}\mathfullstop
    \end{equation*}
    Combining
    \begin{align*}
        \scalarproduct{k}{\nabla^2 u}&=\nabla^i(k_{ij}\nabla^j u)-\nabla^j u \nabla^i k_{ij}\\
        &=\divergence(k(\cdot,\nabla u))-(\divergence{k})(\nabla u)
    \end{align*}
    % We have
    % \begin{align*}
    %     -\int_{\Omega}\scalarproduct{k}{\nabla^2 u}\odif{V}&=\int_{\Omega}
    %     (-\nabla^i(k_{ij}\nabla^j u)+\nabla^j u \nabla^i k_{ij})\odif{V}\\
    %     &=\int_{\Omega}(\divergence{k})(\nabla u)\odif{V}-\int_{\boundary{\Omega}}k(\nu,\nabla u)\odif{A}.
    % \end{align*}
    and
    \begin{align*}
        \divergence((\trace k) \nabla u)&=\nabla_i(\trace{k}) \nabla^i u+(\trace k)\laplacian u\\
        &=\nabla_i((\trace{k})g_{ij})\nabla_j u-(\trace k)^2\abs{\nabla u}\\
        &=-J(\nabla u)+(\divergence{k})(\nabla u)-(\trace k)^2\abs{\nabla u}
    \end{align*}
    gives
    \begin{equation*}
        \int_{\Omega}(J(\nabla u)+(\trace k)^2 \abs{\nabla u}+\scalarproduct{k}{\nabla^2 u})\odif{V}=\int_{\boundary \Omega}(k(\nu,\nabla u)-(\trace k) \scalarproduct{\nu}{\nabla u})\odif{A}\mathfullstop
    \end{equation*}
    Then using
    \begin{equation*}
        \abs{\nabla^2 u}=\abs{\spacetimehessian u}-2\scalarproduct{k}{\nabla^2 u} \abs{\nabla u}-\abs{k}^2\abs{\nabla u}^2
    \end{equation*}
    and the coarea formula yields
    \begin{align*}
        &\int_{\minu}^{\maxu}\int_{S_t} \frac{1}{2}\p*{\frac{\abs{\spacetimehessian u}}{\abs{\nabla u}^2}+R-\abs{k}^2+2 J(\nabla u/\abs{\nabla u})+(\trace k)^2-2K }\odif{A}\odif{t}\\
        &\qquad=\int_{\nonzeroboundary }\p*{\partial_{\nu} \abs{\nabla u}+k(\nu,\nabla u)}\odif{A}\mathfullstop
    \end{align*}
    Applying the definition of \( \mu \) completes the proof.
    
    
\end{proof}
The Riemannian case can be obtained by setting \( k=0 \):
\begin{corollary}[Equality version of \texorpdfstring{}{\nolink}{\cite[Proposition 3.2]{brayHarmonicFunctionsMass2019}}]
    Let \( (\Omega,g,k) \) be an oriented compact initial data set (IDS) with smooth boundary \( \boundary{\Omega} \), having outward unit normal \( \nu \). Let \( u\maps \Omega\to \reals \) be a spacetime harmonic function, and let \( \nonzeroboundary \), \( \minu \), \( \maxu \), \( S_t \) and \( K \) be as in \cref{prop:main_equality}. Then
    \begin{equation*}
        \int_{\nonzeroboundary} (\partial_\nu \abs{\nabla u}+k(\nabla u,\nu))\odif{A}=\int_{\minu}^{\maxu}\int_{S_t} \p*{\frac{1}{2}\frac{\abs{\spacetimehessian u}}{\abs{\nabla u}^2}+\mu+J(\nabla u/\abs{\nabla u})-K}\odif A\odif{t},
    \end{equation*}
    where \( K \) is the level set Gauss curvature.
\end{corollary}
\section{Consequences for mass}
We can now carry this improvement of \cite[Propositon 3.2]{brayHarmonicFunctionsMass2019} through the rest of \cite{brayHarmonicFunctionsMass2019}. We will notice that by just keeping track of one \additionalTerms{additional term} involving the Euler characteristic \( \chi(S_t) \), all the inequalities can be promoted to equalities. In particular we will obtain a new description of the mass of an asymptotically flat manifold instead of the usual lower bound.

In an effort to keep this note mostly self-contained we will repeat pretty much all the computations of \cite{brayHarmonicFunctionsMass2019}. Note that from now on, we will only consider \( 3 \)-dimensional manifolds.
% The proofs of the existence of our harmonic functions and of exterior regions will however not be repeated here, see \cite{brayHarmonicFunctionsMass2019} for those.

\begin{definition}
    We say that a \( 3 \)-dimensional 
    Riemannian manifold \( (M,g) \) 
    % IDS \( (M,g,k) \)
    is asymptotically flat if there exists a compact set \( C \subset M \) such that we can write \( M\setminus C \) as a disjoint union of finitely many \emph{ends} \( M_{\mathrm{end}}^\ell \), where each end is diffeomorphic to the complement of a ball \( \reals^n\setminus B_1 \) (equipped with the standard metric \( \delta \)) and we have the following asymptotic behavior in the \emph{asymptotically flat coordinates} \( x_1,x_2,x_3\) defined by this diffeomorphism:
    \begin{equation}
        \begin{aligned}[t]
            \abs{\partial^l(g_{ij}-\delta_{ij})}&=O(\abs{x}^{-q-l}),\quad l=0,1,2\mathfullstop\\
            % \abs{\partial^l(k_{ij})}&=O(\abs{x}^{-1-q-l}),\quad l=0,1\mathcomma
        \end{aligned}
    \end{equation}
    for some \( q>\frac{n-2}{2} \). We assume that the scalar curvature is integrable, \( R \in L^1(M) \), such that the ADM mass of each end is well-defined and given by
    \begin{equation*}
        m=\lim_{\rho\goesto \infty} 
        %\frac{1}{2(n-1)\omega_{n-1}}
        \frac{1}{16\pi}
        \int_{S_\rho}\sum_{i,j=1}^{n}(g_{ij,i}-g_{ii,j})\nu^j 
        \odif{A}
        % \odif{\mu_{S_\rho}}
        \mathcomma\\
        % P_i&=\lim_{\rho \goesto \infty}
        % %\frac{1}{(n-1)\omega_{n-1}}
        % \frac{1}{8\pi}
        % \int_{S_\rho}\sum_{j=1}^{n}(k_{ij}-(\trace{k})g_{ij})\nu^j \odif{\mu_{S_\rho}},
    \end{equation*}
    where \( \nu \) is the outer unit normal to \( S_\rho \).
    % and \( \omega_{n-1} \) is the volume of the \( (n-1) \)-sphere.
\end{definition}
By \cite[Lemma 4.1]{huiskenInverseMeanCurvature2001}, there exists for each end \( \Mend \) an \emph{exterior region} (\( M_\ext \supset \Mend \)), which is diffeomorphic to the complement of a finite number of open balls (with disjoint closure) in \( \reals^3 \) and has minimal boundary.

Our goal will be to prove the following:
\begin{theorem}[Equality version of \texorpdfstring{}{\nolink}{\cite[Theorem 1.2]{brayHarmonicFunctionsMass2019}}]\label{thm:harmonic_mass_equality}
    Let \( (M_\ext,g) \) be an exterior region of a complete asymptotically flat Riemannian \( 3 \)-manifold \( (M,g) \) with mass \( m \). Then there exists a harmonic function \( u \) on \( (M_\ext,g) \) asymptotic to one of the coordinate functions of the associated end and satisfying Neumann boundary conditions on \( \boundary{M_\ext} \), and we have
    \begin{equation*}
        m\geq \frac{1}{16\pi}\int_{M_{\ext}}\p*{\frac{\abs{\nabla^2 u}^2}{\abs{\nabla u}}+R\abs{\nabla u}}\odif{V}\mathfullstop
    \end{equation*}
    In particular if \( R \) is nonnegative everywhere, then the mass is nonnegative.
\end{theorem}
\begin{proof}[Proof of \cref{thm:harmonic_mass_equality}]
    Consider harmonic coordinates \( x_1,\dotsc,x_n \), see \cite[Section 3.2]{brayHarmonicFunctionsMass2019}. 
\end{proof}
\section{An attempt at generalizing this method to higher dimensions}
\newcommand{\vect}[1]{\mathbf{#1}}
\newcommand{\nestedLevelSet}[1][]{\Sigma_{\ifstrempty{#1}{}{#1,}\vect{t}}} 
\let\nls\nestedLevelSet
The following is a consequence of \cref{prop:main_equality}. We are careless below in always using \( \odif{A} \) instead of whatever measure would be appropriate for the current integration (though we only use it for integration over submanifolds of \( \Omega \), but they need not neccessarily be hypersurfaces). 
\begin{corollary}\label{cor:higher_dim_main_equality_version}
    Let \( u_1,\dotsc,u_n \) be coordinates (in particular, none of the \( u_i \) should have any critical points) on \( (\Omega,g) \) as above, with maxima \( \maxu_i \), minima \( \minu_i \), and level sets \( S_{t}^i  \). Let further \( I_j^k=\prod_{i=j}^{k}\interval{\minu_i}{\maxu_i} \). For \( \vect{t}=(t_1,\dotsc,t_j) \) let \( \nls^j\definedas \bigcap_{i=1}^{j-1} S_{t_i}^i \). Note that \( \nls^1=\Omega \). Define also \( \nls\definedas \nls^{n+1}=\bigcap_{i=1}^n S_{t_i}^i \) (this will, since we consider coordinates, only consist of one point).
    
    Then
    \begin{align*}
        &\int_{I_1^n}\int_{\nls}\frac{1}{2}\p*{\sum_{i=1}^{n}\frac{\abs{\nabla_{\nls^i}^2 u_i}-(\laplacian_{\nls^i}u_i)^2}{\abs{\nabla_{\nls^i}u_i}^2\cdot N_{\vect{t}}}+\frac{R_\Omega}{N_{\vect{t}}}}\odif{A}\odif{\vect{t}}\\
        &\quad=\sum_{i=1}^{n}\int_{I_1^{i-1}}\int_{\boundary{\Omega}\cap \nls^i}\frac{\partial_{\nu_{\nls^i}}\abs{\nabla_{\nls^i}u_i}-\laplacian_{\nls^i}u_i\cdot\frac{\scalarproduct{\nu_{\nls^i}}{\nabla_{\nls^i}u_i}}{\abs{\nabla_{\nls^i}u_i}}}{\prod_{j=1}^{i-1}\abs{\nabla_{\nls^j}u_j}}\odif{A}\odif{\vect{t}}\mathcomma
    \end{align*}
    where \( N_{\vect{t}}=\prod_{i=2}^{n}\abs{\nabla_{\nls^i} u_i} \) and \( \nu_{\nls^i} \) is the normal vector to \( \boundary{\Omega}\cap \nls^i \) inside \( \nls^{i} \).
    % \begin{align*}
    %     &\int_{\Omega}\frac{1}{2}\p*{R_\Omega\cdot\abs{\nabla u_0}\sum_{i=1}^{n}\frac{\abs{\nabla^2 u_i}^2}{\abs{\nabla u_i}^2} }\\
    %     &\quad=\int_{\boundary{\Omega}}\sum_{i=1}^{n}\partial_\nu \abs{\nabla u_i}\odif{A}\mathfullstop
    % \end{align*} 

\end{corollary}
\begin{proof}
    We prove this via induction on the dimension of the space. \cref{prop:main_equality} proves the case \( n=1 \) (after an application of the coarea formula, and noting that the scalar curvature of the level sets is \( 0 \) since they are \( 0 \)-dimensional). Thus assume now that we have proven the claim for \( n-1 \), and let \( \Omega \) be \( n \)-dimensional. 

    We start by considering \cref{prop:main_equality} for \( u=u_1 \),
    \begin{align*}
        &\int_{\minu_1}^{\maxu_1}\int_{S_{t_1}^1}\frac{1}{2}\p*{\frac{\abs{\nabla^2 u_1}^2-(\laplacian u_1)^2}{\abs{\nabla u_1}^2}+R_{\Omega}-R_{S_{t_1}^1}}\odif{A}\odif{t_1}\\
        &\quad=\int_{\boundary{\Omega}}\p*{\partial_\nu \abs{\nabla u_1}-\laplacian u_1\cdot \frac{\scalarproduct{\nu}{\nabla u_1}}{\abs{\nabla u_1}}}\odif{A}\mathfullstop
    \end{align*}
    Since none of the coordinates have any critical points, we can apply the coarea formula \( n-1 \) times, developing in order along \( u_2,\dotsc,u_n \). Using that \( \nabla=\nabla_{\Omega}=\nabla_{\nls^1} \), we obtain
    \begin{equation}\label{eq:higher_dim_main_equality_proposition_applied}
        \begin{aligned}
            &\int_{I_{1}^n}\int_{\nls}\frac{1}{2}\p*{\frac{\abs{\nabla_{\nls^1}^2 u_1}^2-(\laplacian_{\nls^1} u_1)^2}{\abs{\nabla_{\nls^1} u_1}^2\cdot N_{\vect{t}}}+\frac{R_{\Omega}}{N_{\vect{t}}}-\frac{R_{S_{t_1}^1}}{N_\vect{t}}}\odif{A}\odif{\vect{t}}\\
            &\quad=\int_{\boundary{\Omega}\cap \nls^1}\p*{\partial_{\nu_{\nls^1}} \abs{\nabla_{\nls^1} u_1}-\laplacian_{\nls^1} u_1\cdot \frac{\scalarproduct{\nu_{\nls^1}}{\nabla_{\nls^1} u_1}}{\abs{\nabla_{\nls^1} u_1}}}\odif{A}\mathfullstop
        \end{aligned}
    \end{equation}
    Note that \( S_{t_1}^1 \) (for any fixed \( t_1 \)) is \( (n-1) \)-dimensional and has \( u_2,\dotsc,u_n \) as coordinates. Thus we can apply the induction hypothesis to \( S_{t_1}^1 \) and divide by \( \abs{\nabla_{\nls^2} u_2} \) to get
    \begin{align*}
        &\int_{\{t_1\}\times I_2^n}\int_{\nls}\frac{1}{2}\p*{\sum_{i=2}^{n}\frac{\abs{\nabla_{\nls^i}^2 u_i}-(\laplacian_{\nls^i}u_i)^2}{\abs{\nabla_{\nls^i}u_i}^2\cdot N_{\vect{t}}}+\frac{R_{S_{t_1}^1}}{N_{\vect{t}}}}\odif{A}\odif{\vect{t}}\\
        &\quad=\sum_{i=2}^{n}\int_{I_2^{i-1}}\int_{\boundary{\Omega}\cap \nls^i}\frac{\partial_{\nu_{\nls^i}}\abs{\nabla_{\nls^i}u_i}-\laplacian_{\nls^i}u_i\cdot\frac{\scalarproduct{\nu_{\nls^i}}{\nabla_{\nls^i}u_i}}{\abs{\nabla_{\nls^i}u_i}}}{\prod_{j=2}^{i}\abs{\nabla_{\nls^j}u_j}}\odif{A}\odif{\vect{t}}\mathcomma
    \end{align*}
    Integrating this over \( t_1\in \interval{\minu_1}{\maxu_1} \) and adding it to \cref{eq:higher_dim_main_equality_proposition_applied} concludes the induction step.
\end{proof}
Note in particular that in the above corollary, we get rid (via a telescoping sum) of any level set scalar curvature term (which was problematic in the generalization of the harmonic level set method to higher dimensions). Using the above corollary, we might thus hope to be able to apply a variant of the method in arbitrary dimension in the case where we have global \emph{nested harmonic coordinates} (defined below) on all of \( M \). 

\begin{definition}
    We say \( u_1,\dotsc,u_m \) are \emph{nested harmonic}, if we have for \( \nls^i \) as above (in \cref{cor:higher_dim_main_equality_version})
    \begin{equation*}
        \laplacian_{\nls^i}u_i=0\mathcomma
    \end{equation*}
    i.e.~\( u_1 \) is harmonic on \( \Omega \), \( u_2 \) is harmonic on the level sets of \( u_1 \), and \( u_i \) in general is harmonic on the intersection of the level sets of the previous nested harmonic functions.

    If \( u_1,\dotsc,u_n \) are nested harmonic and also provide a coordinate chart of \( \Omega \), we say they are \emph{nested harmonic coordinates}.
\end{definition}
\begin{corollary}\label{cor:harmonic_higher_dim_main_equality_version}
    Let \( u_1,\dotsc,u_n \) be nested harmonic coordinates. Then with the notation fixed above we have
    \begin{align*}
        &\int_{I_1^n}\int_{\nls}\frac{1}{2}\p*{\sum_{i=1}^{n}\frac{\abs{\nabla_{\nls^i}^2 u_i}}{\abs{\nabla_{\nls^i}u_i}^2\cdot N_{\vect{t}}}+\frac{R_\Omega}{N_{\vect{t}}}}\odif{A}\odif{\vect{t}}\\
        &\quad=\sum_{i=1}^{n}\int_{I_1^{i-1}}\int_{\boundary{\Omega}\cap \nls^i}\frac{\partial_{\nu_{\nls^i}}\abs{\nabla_{\nls^i}u_i}}{\prod_{j=1}^{i-1}\abs{\nabla_{\nls^j}u_j}}\odif{A}\odif{\vect{t}}\mathfullstop
    \end{align*}
\end{corollary}
\begin{remark}
    If it is possible to find nested harmonic functions exhibiting some bounding behavior like
    \begin{equation*}
        \frac{ \abs{\nabla^2_{\nls^i} u_1}}{\abs{\nabla_{\nls^i} u_1}^2}-R_{\nls^i}\geq -C\cdot \abs{\nabla u_j}
    \end{equation*}
    for any \( j>i \) even in the presence of critical points, then I believe one could show a version of \cref{cor:harmonic_higher_dim_main_equality_version} for any such set of harmonic functions. One might hope that one can find (on any given asymptotically flat manifold) such a set of functions that also forms an asymptotically flat coordinate system. This feels like it might be a (very hazy) path towards proving the positive mass theorem in arbitrary dimension.
\end{remark}
\subsection{Some calculations towards maybe showing a positive mass theorem}
The nested harmonic functions below  definitely do not exist on any manifold that is not diffeomorphic to \( \reals^n \), so keep in mind that all the calculations here are not particularly interesting without some way to generalize this.

Let \( (M,g) \) be diffeomorphic to \( \reals^n \) and asymptotically flat with integrable scalar curvature \( R \). Assume we have nested harmonic asymptotically flat coordinates \( u_1,\dotsc,u_n \) (none of which have any critical points) and let \( C_L \) be the coordinate cube of side length \( 2 L \), centered at the origin of our coordinates. Let \( \Omega_L \) be the closure of the compact component of \( M\setminus C_L \). Let also \( F_{\pm L,i} \) denote the face of \( C_L \) with constant \( u_i=\pm L \), such that \( C_{L}=\bigcup_{i=1}^n F_{+L,i}\cup F_{-L,i} \).

Recall that the mass is then given by
\begin{equation*}
    m=\lim_{L \goesto \infty}\frac{1}{2(n-1)\omega_{n-1}}\int_{C_L}\sum_{i,j=1}^{n}(g_{ij,i}-g_{ii,j})\odif{A}\mathfullstop
\end{equation*}
On the other hand we obtain from \cref{cor:harmonic_higher_dim_main_equality_version}
\begin{equation}
    \begin{aligned}
        &\int_{I_1^n}\int_{\nls[L]}\frac{1}{2}\p*{\sum_{i=1}^{n}\frac{\abs{\nabla_{\nls^i}^2 u_i}}{\abs{\nabla_{\nls^i}u_i}^2\cdot N_{\vect{t}}}+\frac{R_\Omega}{N_{\vect{t}}}}\odif{A}\odif{\vect{t}}\\
        &\quad=\sum_{i=1}^{n}\int_{I_1^{i-1}}\int_{C_L\cap \nls^i}\frac{\partial_{\nu_{\nls^i}}\abs{\nabla_{\nls^i}u_i}}{\prod_{j=1}^{i-1}\abs{\nabla_{\nls^j}u_j}}\odif{A}\odif{\vect{t}}\mathcomma
    \end{aligned}
\end{equation}
where \( \nls[L] \definedas \Omega_L \cap \nls=\Omega_L\cap \{u_i=t_i\quad \forall i\} \).

Let us now compute the boundary terms:
\begin{lemma}\label{lem:boundary_computation}
    In the notation fixed above, we have \todo[fill in result]
    \begin{equation*}
        \int_{I_1^{i-1}}\int_{C_L\cap \nls^i}\frac{\partial_{\nu_{\nls^i}}\abs{\nabla_{\nls^i}u_i}}{\prod_{j=1}^{i-1}\abs{\nabla_{\nls^j}u_j}}\odif{A}\odif{\vect{t}}=???\mathfullstop
    \end{equation*}
\end{lemma}
We first calculate the asymptotic behavior of some quantities. Let from now on \( n_{\nls^i} \) be the normal vector to \( \nls^i \) inside \( \nls^{i+1} \).
\begin{lemma}
    For all \( i=1,\dotsc,n \), and \( \abs{x}=L \) we have
    \begin{equation*}
        \abs{\nabla_{\nls^i}u_i}^2=g^{ii}+O_2(\abs{x}^{-2q})
    \end{equation*}
    and 
    \begin{equation*}
        n_{\nls^i}=\partial_i+O_2(\abs{x}^{-q}).
    \end{equation*}
\end{lemma}
\begin{proof}
    Note first that for all \( i \), the normal to the level set \( S_{t_i}^i \)  has coordinates 
    \begin{equation*}
        g^{jk}\delta_{ik}=\delta_{ik}+O_2(\abs{x}^{-q})
    \end{equation*} 
    (since it is equal to \( (\odif{u_i})^\sharp \), and \( \odif{u_i} \) has coordinates \( \delta_{ik} \)). In particular, the second claim is true for \( n=1 \).

    Note also that \( \abs{\nabla u_i}^2=\abs{\odif{u_i}}^2=g^{ii}=\delta^{ii}+O(\abs{x}^{-q}) \) (by a Taylor expnasion of the matrix inverse).
    
    We proceed further by induction. Assume the second claim is true for all \( i\leq j \). Then we obtain on one hand
    \begin{equation*}
        \abs{\nabla_{\nls^j}u_j}^2=\abs{\nabla u_j}^2-\sum_{k=1}^{j}\scalarproduct{n_{\nls^k}}{\nabla u_j}^2=g^{jj}+O_2(\abs{x}^{-2q})\mathcomma
    \end{equation*}
    since
    \begin{equation*}
        \scalarproduct{n_{\nls^k}}{\nabla u_j}=\abs{\nabla u_j}\cdot (\scalarproduct{\partial_k}{\partial_j}+O_2(\abs{x}^{-q}))=O_2(\abs{x}^{-q})\mathfullstop
    \end{equation*}
    On the other hand we get
    \begin{align*}
        n_{\nls^{j+1}}&=\partial_{j+1}+O_2(\abs{x}^{-q})-\sum_{k=1}^{j}n_{\nls^j}\cdot \scalarproduct{n_{\nls^k}}{\partial_{j+1}+O_2(\abs{x}^{-q})}\\
        &=\partial_{j+1}+O_2(\abs{x}^{-q}).
    \end{align*}
\end{proof}
\begin{proof}[Proof of \cref{lem:boundary_computation}]
    To begin, note that 
    \begin{equation*}
        \nabla_{\nls^i}\abs{\nabla_{\nls^i}}=\nabla(g^{ii}+O_2(\abs{x}^{-2q}))^{1/2}=-\frac{1}{2}\nabla_{\nls^i}g_{ii}+O_1(\abs{x}^{-1-2q})
    \end{equation*}
    and
    \begin{equation*}
        \evaluateat{\nu_{\nls^i}}{F_{\pm i,L}}=\pm \partial_i+O_2(\abs{x}^{-q})-\sum_{j=1}^{i-1}n_{\nls^j}\scalarproduct{n_{\nls^j}}{\pm \partial_i+O_2(\abs{x}^{-q})}=\pm\partial_i+O_2(\abs{x})\mathfullstop
    \end{equation*}
    We further have
    \begin{equation*}
        \frac{1}{\prod_{j=1}^{i-1}\abs{\nabla_{\nls^j}u_j}}=\frac{1}{\prod_{j=1}^{i-1}(\delta^{ii}+O_2(\abs{x}^{-q}))}=1+O_2(\abs{x}^{-q})\mathfullstop
    \end{equation*}
    % and similarly
    % \begin{equation*}
    %     \prod_{j=1}^{i-1}\abs{\nabla u_j}=1+O_2(\abs{x}^{-q}).
    % \end{equation*}
    It follows that
    \begin{equation*}
        \int_{I_1^{i-1}}\int_{F_{\pm L,j}\cap \nls^i}\frac{\partial_{\nu_{\nls^i}}\abs{\nabla_{\nls^i}u_i}}{\prod_{j=1}^{i-1}\abs{\nabla_{\nls^j}u_j}}\odif{A}\odif{\vect{t}}=\int_{I_1^{i-1}}\int_{ F_{\pm L,j}\cap \nls^i}\pm g_{ii,j}\odif{A}\odif{\vect{t}}+O_2(L^{1-2q})\mathcomma
    \end{equation*}
    where we have used that \( \nabla_{\nls^i}g_{ii}=O_1(\abs{x}^{-1-q}) \). Note that the above is zero (!) if \( j<i \) (as then for almost all \( \vect{t} \) we have \( F_{\pm L,j}\cap \nls^i=\emptyset \)). Note also that
    \begin{equation*}
        1-2q\leq 1-(n-2)<0
    \end{equation*} 
    for all \( n\geq 3 \).


\end{proof}


\endgroup

\printbibliography
\end{document}